%!TEX program = pdflatex
\documentclass[11pt, a4paper]{article}
\usepackage{amsfonts, setspace, mathptmx, bm, amsmath, color, amssymb, amsthm, mathrsfs}
\renewcommand{\baselinestretch}{1.5}
\begin{document}
\title{Calculus Note}
\author{Pan Hao, Henry}
\maketitle

    \section{Euclidean n-space}
        \subsection{Concepts:}
            $$\begin{aligned}
                \mathbb{R}^n &:= \{(x_1, ..., x_n)|x_1, ..., x_n \in \mathbb{R}\} \\
                A \times B &:= (a \in A, b \in B), \mathbb{R}^n := \mathbb{R} \times \mathbb{R} \times ... \times \mathbb{R} \\
                ||\cdot|| &:= \mathbb{R}^n \rightarrow \mathbb{R} \text{ (Euclidean n-norm)} \\
                ||\vec{x}|| &:= \sqrt{x_1^2 + x_2^2 + ... + x_n^2} \\
                d &:= \mathbb{R}^n \times \mathbb{R}^n \rightarrow \mathbb{R} \text{ (Distance)} \\
                d(\vec{x}, \vec{y}) &:= ||\vec{x} - \vec{y}|| = \sqrt{(x_1 - y_1)^2 + ... + (x_n - y_n)^2} \\
                &\text{Properties of distance:} \\
                \text{Symmetry:}& \forall \vec{x}, \vec{y} \in \mathbb{R}^n, d(\vec{x}, \vec{y}) = d(\vec{y}, \vec{x}) \\
                \text{Triangle inequality:}& \forall \vec{x}, \vec{y}, \vec{z} \in \mathbb{R}^n, d(\vec{x}, \vec{y}) + d(\vec{y}, \vec{z}) \geq d(\vec{x}, \vec{z}) \\
                \text{Positivity:}& \forall \vec{x}, \vec{y} \in \mathbb{R}^n, d(\vec{x}, \vec{y}) \geq 0 \\
                \Rightarrow& \forall \vec{x}, \vec{y} \in \mathbb{R}^n, d(\vec{x}, \vec{y}) = 0 \Leftrightarrow \vec{x} = \vec{y}
            \end{aligned}$$
        \subsection{Open n-ball}
            \subsubsection{Definition:} For $\delta > 0, B(\vec{x}, \delta) := \{\vec{y} \in \mathbb{R}^n | d(\vec{x}, \vec{y}) < \delta\}$
            \subsubsection{Interior points:} $\vec{x} \in \mathbb{R}^n$ is an interior point of $S$ iff. $\exists \delta > 0 s.t. B(\vec{x}, \delta) \subset S$
            \subsubsection{Exterior points:} $\vec{x} \in \mathbb{R}^n$ is an exterior point of $S$ iff. $\exists \delta > 0 s.t. B(\vec{x}, \delta) \subset \mathbb{R}^n \setminus S$
            \subsubsection{Boundary points:} $\vec{x} \in \mathbb{R}^n$ is an boundary point of $S$ iff. $\forall \delta > 0, B(\vec{x}, \delta) \cap S \neq \varnothing \text{ and} B(\vec{x}, \delta) \cap (\mathbb{R}^n \setminus S) \neq \varnothing$
            \subsubsection{Adherent points:} $\vec{x} \in \mathbb{R}^n$ is an adherent point of $S$ iff. $\forall \delta > 0, B(\vec{x}, \delta) \cap S \neq \varnothing$
            \subsubsection{Limit points:} $\vec{x} \in \mathbb{R}^n$ is an limit point of $S$ iff. $\forall \delta > 0, (B(\vec{x}, \delta) \setminus \{\vec{x}\}) \cap S \neq \varnothing$
        \subsection{Type of sets}
        \subsubsection{Open sets:} A set S is open iff. every point of S is an interior point.
        \subsubsection{Closed sets:} A set S is closed iff. $\mathbb{R}^n \setminus S$ is open.
        \subsubsection{Interior of a set:} All interior points of S.
        \subsubsection{Exterior of a set:} All exterior points of S.
        \subsubsection{Boundary of a set} All boundary points of S $\rightarrow \partial S$
        \subsubsection{Closure of a set:} $\bar{S} \text{ of } S \text{ is } S \cup \partial S$
    \subsection{Theorem 1.1}
    A set $S \subset \mathbb{R}^n$ is open iff. it is equal to the union of a collection of open balls. \\
        \textbf{Proof: 1. $S$ = union of open balls $\Rightarrow S =$ open}  \\
        Consider arbitrary $\vec{x} \in S$, then $\exists$ an open ball $B(\vec{y}, \delta_y) \subset S$ which contains $\vec{x}$. Let $\delta_{\vec{x}} := \delta_{\vec{y}} -d(\vec{x}, \vec{y}), \text{ then } \delta_{\vec{x}} > 0 \text{ and } B(\vec{x}, \delta-{\vec{x}}) \subset B(\vec{y}, \delta_{\vec{y}}) \subset S \\ \Rightarrow \vec{x} \text{ is an interior point of } S$. \\
        Since $\vec{x}$ was arbitrary chosen, every point of $S$ is an interior point. Therefore, $S$ is open. \\
        \textbf{2. $S =$ open $\Rightarrow S =$ union of open balls} \\
        Assume $S$ is open, $\Rightarrow \forall \vec{x} \in S, \exists \delta_{\vec{x}} > 0 s.t. B(\vec{x}, \delta_{\vec{x}}) \subset S$ \\
        $$U := \bigcup\limits_{\vec{x} \in \mathrm{Int}(S)} B(\vec{x}, \delta_{\vec{x}}) \subset S$$ \\
        Conversely, $S \subset U \Rightarrow S = U$
        Therefore, S is the union of open balls.
    \subsection{Theorem 1.2}
    $\text{Int}(S)$ of any set $S \subset \mathbb{R}^n$ is an open set. Likewise, $\text{Ext}(S) \text{ of } S$ is also open. \\
        \textbf{Proof:} Every $\vec{x} \in \text{Int}(S)$ is an interior point of $S \\ \Rightarrow \exists \delta_{\vec{x}} > 0, s.t. B(\vec{x}, \delta_{\vec{x}}) \subset S \\
        \Rightarrow U := \bigcup\limits_{\vec{x} \in \mathrm{Int}(S)} B(\vec{x}, \delta_{\vec{x}})$ is open. \\
        Thus, $\text{Int}(S) \subset U$, because every $\vec{x} \in \text{Int}(S)$ is contained in some open ball \\ $B(\vec{x}, \delta_{\vec{x}}) \subset U$. For every $\vec{x} \in \text{Int}(S), B(\vec{x}, \delta_{\vec{x}}) \subset \text{Int}(S)$. \\
        Since $B(\vec{x}, \delta_{\vec{x}})$ is open, every point in $B(\vec{x}, \delta_{\vec{x}})$ is an interior point of $B(\vec{x}, \delta_{\vec{x}}) \Rightarrow$ is an interior point of $S \supset B(\vec{x}, \delta_{\vec{x}})$.
    \subsection{Proposition 1.3}
    The union of any collection of (arbitrarily many) open sets is open.
    \subsection{Proposition 1.4}
    Finite intersections of open sets is open.
    \subsection{Theorem 1.5}
    Finite unions of closed sets is closed, and intersections of closed sets is closed.
\section{Further properties}
    \subsection{Proposition 2.1}
    Finite intersections of open sets is open.
    \subsection{Theorem 2.2}
    Finite unions of closed set is closed, and intersections of closed set is closed. \\
        \textbf{Proof:} 
        $$\begin{aligned}
            &\mathbb{R}^n \setminus (\bigcup\limits_{\alpha \in \mathscr{A}} S_\alpha) = \bigcup\limits_{\alpha \in \mathscr{A}} (\mathbb{R}^n \setminus S_\alpha) \\
            \Rightarrow& \vec{x} \in \mathbb{R}^n \setminus (\bigcup\limits_{\alpha \in \mathscr{A}} S_\alpha) \\
            \Leftrightarrow& \vec{x} \notin \bigcup\limits_{\alpha \in \mathscr{A}} S_\alpha \\
            \Leftrightarrow& \forall \alpha \in \mathscr{A}, \vec{x} \notin S_\alpha \\
            \Leftrightarrow& \forall \alpha \in \mathscr{A}, \vec{x} \in \bigcap\limits_{\alpha \in \mathscr{A}} (\mathbb{R}^N \setminus S_\alpha)
        \end{aligned}$$
    \subsection{Connectedness}
        \subsubsection{Path-connectedness:}
            $\forall \vec{x}, \vec{y} \in S, \exists$ a continuous function $f: [0, 1] \to S, s.t. f(0) = \vec{x} \text{ and } f(1) = \vec{y}$
        \subsubsection{Connectedness:}
        $S \subset \mathbb{R}, s.t. \exists \text{ open sets } U, V \subset \mathbb{R}^n, s.t. U \cap V = \varnothing, U, V \neq \varnothing \\
        S \subset U \cup V$ and $S \cap V$ and $S \cup V$ are both non-empty.
    \subsection{Convergence}
        \subsubsection{Definition}
            Let $\{\vec{x_k}\}_{k \in \mathbb{N}} = {\vec{x_1}, \vec{x_2}, ...}$ be a sequence in $\mathbb{R}^n$, we say that this sequence converges to $\vec{a} \in \mathbb{R}^n$ iff. $\forall \varepsilon > 0, \exists N \in \mathbb{N}, s.t. \forall k > N, ||\vec{x_k} - \vec{a}|| < \varepsilon \\
            \Rightarrow \lim\limits_{k \to \infty} \vec{x_k} = \vec{a} \Leftrightarrow d(\vec{x_k}, \vec{a}) = 0$
        \subsubsection{Cauchy convergent sequences}
        A sequence $\{\vec{x_k}\} \text{ in } \mathbb{R}^n$ is called Cauchy convergent iff. $\forall \varepsilon > 0, \exists N, s.t. \forall j, k > N, d(\vec{x}, \vec{k}) < \varepsilon$
    \subsection{Theorem 2.4}
    Let $\{\vec{x_k}\}$ be a sequence in $\mathbb{R}^n$, we adopt the following notion: 
    $$\vec{x_k} = (x_{k, 1}, x_{k, 2}, ..., x_{k_n})$$
    Then $\lim\limits_{k \to \infty} \vec{x_k} = \vec{a} \Leftrightarrow \forall j = 1, ..., n, \lim\limits_{k \to \infty} x_{k, j} = a_j$ \textbf{Similar to Cauchy convergence} \\
    \textbf{N.B.} In $\mathbb{R}^n$, Cauchy convergent \textbf{doesn't mean} convergent in $S$ ($\partial S$).
    \subsection{Lemma 2.5}
        $S \subset \mathbb{R}^n, \vec{a} \in \mathbb{R}^n, \vec{a} \text{ is adherent point of } S \Leftrightarrow \{\vec{x_k}\} \text{ in } S \text{ converge to } \vec{a}$
    \subsection{Lemma 2.6}
        $\vec{a}$ is a limit point $\Leftrightarrow \{\vec{x_k}\} \text{ in } S \setminus \{\vec{a} \}$ converges to $\vec{a}$
    \subsection{Theorem 2.7}
    $S \subset \mathbb{R}^n \text{ is closed} \Leftrightarrow S$ contains all of its adherent points.
    \subsection{Diameter of a set}
        \subsubsection{Definition}
        $diam(S) := \sup\limits_{\vec{x}, \vec{y} \in S} d(\vec{x}, \vec{y}) \to R \in (0, +\infty) \cup \{+\infty\}$
        \subsubsection{Bounded sets and sequences}
        $diam(S) < \infty$
    \subsection{Cantor's intersection theorem}
    $\{F_k \}_{k \in \mathbb{N}}$ of non-empty, closed \textit{(without this is not true)} subsets of $\mathbb{R}^n$ nested in the following way:
    $$\mathbb{R}^n \supset F_1 \supset F_2 \supset ... \supset F_k \supset ...$$
    If $\lim\limits_{k \to \infty} diam(F_k) = 0$, then the $\bigcup\limits_{k = 1}^{\infty} F_k$ is a set consisting a single element. 
\section{Vector function}
    \subsection{Function}
    $f$ is a subset of $A \times B:$
    $$\forall a \in A, \exists! b \in B, s.t. (a, b) \in f \Leftrightarrow f(a) = b$$
    \subsection{Scalar and vector valued functions}
        \subsubsection{Definition}
        $f: \Omega_1 \to \Omega_2$ between a set $\Omega_1 \subset \mathbb{R}^n \text{ and } \Omega_2 \subset \mathbb{R}^m$ is called an \textbf{n-variable, m-dimensioned vector valued} function. Especially when $m = 1$, it is a scalar function.
        \subsubsection{Algebraic constructions}
        $$\begin{aligned}
            &\text{For } \Omega \subset \mathbb{R}^n, \mu: \Omega \to \mathbb{R}, \bm{f}, \bm{g}: \Omega \to \mathbb{R}^n: \\
            &\mu\bm{f}: \Omega \to \mathbb{R}^n, \vec{x} \mapsto \mu(\vec{x})\bm{f}(\vec{x}) \\
            &\bm{f} + \bm{g}: \Omega \to \mathbb{R}^n, \vec{x} \mapsto \bm{f}(\vec{x}) + \bm{g}(\vec{x})
        \end{aligned}$$
        \subsubsection{Composition}
        $$\begin{aligned}
            &\bm{f}: \Omega_1 \subset \mathbb{R}^n \to \Omega_2 \subset \mathbb{R}^m, \bm{g}: \Omega_2 \to \mathbb{R}^l := \\
            &\bm{g} \circ \bm{f}: \Omega_1 \to \mathbb{R}^l, \vec{x} \mapsto \bm{g}(\bm{f}(\vec{x})) \\
            &\text{Practically requires } \bm{f}(\Omega_1) = {\bm{f}(\vec{x}) | \vec{x} \in \Omega_1} \text{ is a subset of the domain of } \bm{g}
        \end{aligned}$$
    \subsection{Taking limits}
    Let $\Omega \subset \mathbb{R}^n$ be non-empty. Given a function $\bm{f}: \Omega \to \mathbb{R}^n$, a limit point $\vec{x_0}$ of $\Omega$ and $\vec{a} \in \mathbb{R}^m$. If
    $$\begin{aligned}
        &\forall \varepsilon > 0, \exists \delta > 0, s.t. \forall \vec{x} \in \Omega \\
        &0 < ||\vec{x} - \vec{x_0}||_n < \delta \Rightarrow ||\bm{f}(\vec{x}) - \vec{a}||_m < \varepsilon \\
        &\Leftrightarrow \vec{x} \in B(\vec{x_0}, \delta) \setminus \{\vec{x_0} \} \Rightarrow \bm{f}(\vec{x}) \in B(\vec{a}, \varepsilon)
        \Rightarrow \vec{x} \text{ limits to } \vec{x_0} \text{ in } \Omega. \bm{f}(\vec{x}) \text{ limits to } \vec{a}: \\
        &\lim\limits_{\Omega \ni \vec{x} \to \vec{x_0}} \bm{f}(\vec{x}) = \vec{a} \\
    \end{aligned}$$
    When limit point $\vec{x_0}$ is an interior point of $\Omega \cup \{\vec{x_0} \}$, then we can denote
    $$\lim\limits_{\vec{x} \to \vec{x_0}} \bm{f}(\vec{x})$$
    Coordinate-wise denotation:
    $$\lim\limits_{\Omega \ni \vec{x} \to \vec{x_0}} \bm{f}(\vec{x}) = \vec{a} \Leftrightarrow \text{For }j = 1, ..., m, \lim\limits_{\Omega \ni \vec{x} \to \vec{x_0}} f_j(\vec{x}) = a_j, \bm{f}(\vec{x}) = (f_1{\vec{x}, ..., f_n(\vec{x})})$$
    \subsection{Scalar function properties}
        \subsubsection{Inequality}
        If $\forall \vec{x} \in \Omega \setminus \{\vec{x_0} \}, f(\vec{x}) \leq g(\vec{x})$, then
        $$\lim\limits_{\Omega \ni \vec{x} \to \vec{x_0}} f(\vec{x}) \leq \lim\limits_{\Omega \ni \vec{x} \to \vec{x_0}} g(\vec{x})$$
        \subsubsection{Non-negativity and Sandwich Theorem}
        \subsubsection{Calculation}
        If $\lim\limits_{\Omega \ni \vec{x} \to \vec{x_0}} \mu(\vec{x}), \bm{f}(\vec{x}), \bm{g}(\vec{x})$ all exist, then
        $$\begin{aligned}
            \lim\limits_{\Omega \ni \vec{x} \to \vec{x_0}} (\mu\bm{f})(\vec{x}) &= \lim\limits_{\Omega \ni \vec{x} \to \vec{x_0}} \mu(\vec{x}) \cdot \bm{f}(\vec{x}) \\
            \lim\limits_{\Omega \ni \vec{x} \to \vec{x_0}}(\bm{f} + \bm{g})(\vec{x}) &= \lim\limits_{\Omega \ni \vec{x} \to \vec{x_0}}\bm{f}(\vec{x}) + \lim\limits_{\Omega \ni \vec{x} \to \vec{x_0}}\bm{g}{\vec{x}} \\
            \lim\limits_{\Omega \ni \vec{x} \to \vec{x_0}}\frac{1}{\mu(\vec{x})} &= \frac{1}{\lim\limits_{\Omega \ni \vec{x} \to \vec{x_0}}\mu(\vec{x})} (\mu \neq 0)
        \end{aligned}$$
    \subsection{Uniqueness}
    If a limit exists, then it must be unique. \\
        \textbf{Proof:} Given $\lim\limits_{\Omega \ni \vec{x} \to \vec{x_0}} \bm{f}(\vec{x}) = \vec{a}, \vec{b}$,
        $$\begin{aligned}
            &\Rightarrow \text{A sequence } \{\vec{x_k}\}_{k \in \mathbb{N}} \text{ has its tail } \to \vec{a}, \vec{b} \\
            &\Rightarrow \vec{a}, \vec{b} \text{ are arbitrary close \textit{(via triangle inequality)}} \\
            &\Rightarrow \vec{a} = \vec{b}
        \end{aligned}$$
    \subsection{Iterated limits / Repeated limits}
    Taking limits of a function one variable at a time.
    $$\lim_{x \to x_0}\lim_{y \to y_0} f(x, y)$$
    1. Compute $\lim\limits_{y \to y_0} f(x, y)$ for all $x$ in a small punctured ball $B(x_0, \delta) \setminus \{x_0\}$ around $x_0$ \\
    2. This produces a function $\varphi(x) := \lim\limits_{x \to x_0} f(x, y)$ on $B(x_0, \delta) \setminus \{x_0 \} \Rightarrow \lim\limits_{x \to x_0} \varphi(x)$
    $$\Rightarrow \lim_{x \to x_0}\lim_{y \to y_0} f(x, y) = \lim\limits_{x \to x_0} \varphi(x) = \lim_{(x, y) \to (x_0, y_0)} f(x, y)$$
    If $$\lim_{x \to x_0}\lim_{y \to y_0} f(x, y) \neq \lim_{y \to y_0}\lim_{x \to x_0} f(x, y)$$
    then $\lim_{(x, y) \to (x_0, y_0)} f(x, y)$ doesn't exist. \\
    \textbf{Conclusion:} Iterated limits $\nLeftrightarrow$ double limits
\section{Continuity}
    \subsection{Continuity}
        \subsubsection{Definition}
        $$\begin{aligned}
            \bm{f}: \Omega \subset \mathbb{R}^n \to \mathbb{R}^m, \vec{x_0} \in \Omega, \bm{f} \text{ is continuous at } \vec{x_0} &\Leftrightarrow \\
            \forall \varepsilon > 0, \exists \delta > 0, s.t. \forall \vec{x} \in \Omega, ||\vec{x} - \vec{x_0}|| < \delta &\Rightarrow ||\bm{f}(\vec{x}) - \bm{f}(\vec{x_0})|| < \varepsilon \\
            (\vec{x} \in B(\vec{x_0}, \delta) \cap \Omega &\Rightarrow \bm{f}(\vec{x}) \in B(\bm{f}(\vec{x_0}), \varepsilon))
        \end{aligned}$$
        If $\vec{x_0} \in \Omega$ is isolated, then $\bm{f}$ is always continuous at $\vec{x_0}$. \\
        if $\vec{x_0} \in \Omega$ is a limit point ,then Continuity at $\vec{x_0}$ is equal to the following limit being true: $\lim\limits_{\Omega \ni \vec{x} \to \vec{x_0}}\bm{f}(\vec{x}) = \bm{f}(\vec{x_0})$
        \subsubsection{Continuous function}
        $\bm{f}: \Omega \to \mathbb{R}^m$ is continuous at every $\vec{x} \in \Omega := C^0$ \\
        Scalar function: $C^0(\Omega) := C^0(\Omega; \mathbb{R})$ \\
        $\bm{f}(\vec{x})$ is continuous $\Leftrightarrow f_1(\vec{x}), f_2(\vec{x}), ..., f_m(\vec{x})$ are continuous.
    \subsection{Theorem 4.1}
    Given continuous functions $\bm{f}: \Omega_1 \subset \mathbb{R}^n \to \mathbb{R}^m, \bm{g}: \Omega_2 \subset \mathbb{R}^m \to \mathbb{R}^l$ \\
    such that $\bm{f}(\Omega_1) \subset \Omega_2$, then the function $\bm{g} \circ \bm{f}: \Omega_1 \to \mathbb{R}^l$ is continuous.
    \subsection{Theorem 4.2}
    Given a function $\bm{f}: \Omega \to \mathbb{R}^m$, where $\Omega$ is an open set. Then $\bm{f}$ is continuous iff. for every open set $U \subset \mathbb{R}^m$, the preimage set: $\bm{f}^{-1}(U) := \{\vec{x} \in \Omega | \bm{f}(\vec{x} \in U)\}$ is open. \\
    If $\Omega$ isn't open, then $\bm{f}$ is open iff. for all open sets $U \subset \mathbb{R}^m$, the preimage set $\bm{f}^{-1}(U)$ is the intersection of $\Omega$ with an open set in $\mathbb{R}^n \supset \Omega$
    \subsection{Extreme value theorem}
    Given closed, bounded set $\Omega \subset \mathbb{R}^n$, continuous scalar function $f \in C^0(\Omega)$, then $f$ attains its maximum/minimum on $\Omega$: \\
    $\Rightarrow$ There exists $m, M > 0$ s.t. $\forall \vec{x} \in \mathbb{R}^n$,
    $$m\sum\limits_{j = 1}^n|x_j| \leq ||\vec{x}|| \leq M\sum\limits_{j = 1}^n|x_j|$$
    \subsection{Differentiability}
    $\bm{f}: \Omega \subset \mathbb{R}^n \to \mathbb{R}^m$ is differentiable at an interior point $\vec{x} \in \Omega$ iff. there exists $\mathbb{L}: \mathbb{R}^n \to \mathbb{R}^m$ s.t.
    $$\lim\limits_{\vec{h} \to 0} \frac{||\bm{f}(\vec{x} + \vec{h}) - \bm{f}(\vec{x}) - \mathbb{L}(\vec{h})||_m}{||\vec{h}||_n} = 0$$
   We denote $\mathbb{L}$ by $\mathrm{d}\bm{f}|_{\vec{x}}$ or $\mathrm{d}\bm{f}(\vec{x})$, it is the derivative of $\bm{f}$ at $\vec{x}$. When $\Omega$ is open, $\bm{f}$ is differentiable iff. $\bm{f}$ is differentiable of every $\vec{x} \in \Omega$. \\
   \textbf{Differentiability implies continuity}
   \subsection{Theorem 4.3} 
   If $\bm{f}: \Omega \to \mathbb{R}^m$ is differentiable at $\vec{x}$, then the derivative $\mathrm{d}\bm{f}|_{\vec{x}}$ at $\vec{x}$ is unique.
   \subsection{Partially differentiate functions}
   $$\frac{\partial f_i}{\partial x_j}(\vec{t}) := \frac{\partial f_i}{\partial x_j}|_{\vec{t}} := \lim\limits_{h \to 0}\frac{f_i(t_1, ..., t_{j - 1}, t_j + h, t_{j + 1}, ..., t_n) - f_i(\vec{t})}{h}$$
   \subsubsection{Matrix expression for the total derivative}
   $$\begin{aligned}
       \begin{bmatrix}
           \mathrm{d}\bm{f}|_{\vec{t}}
       \end{bmatrix} = 
       \begin{bmatrix}
           \frac{\partial f_1}{\partial x_1}|_{\vec{t}} & \ldots & \frac{\partial f_1}{\partial x_n}|_{\vec{t}} \\
           \vdots & & \vdots \\
           \frac{\partial f_m}{\partial x_1}|_{\vec{t}} & \ldots & \frac{\partial f_m}{\partial x_n}|_{\vec{t}}
       \end{bmatrix}
   \end{aligned}$$
   \textbf{Jacobian} of $\bm{f}$ of $\vec{t}$
   \subsubsection{Little o notation}
   Given scalar functions $f, g: \Omega \to \mathbb{R}$, if
   $$\lim\limits_{\Omega \ni \vec{x} \to \vec{x_0}} \frac{|\bm{f}(\vec{x})|}{|\bm{g}(\vec{x})|} = 0$$
   We can write:
   $$||\bm{g}(\vec{h})|| = o(||\vec{h}||) (\vec{h} \to \vec{0})$$
   \subsubsection{Big O notation}
   Given scalar functions $f, g: \Omega \to \mathbb{R}$, $\exists C, \delta > 0$,
   $$\forall \vec{x} \in B(\vec{x_0}, \delta) \setminus \{\vec{x_0}\} \cap \Omega, |\bm{f}(\vec{x})| \leq C|\bm{g}(\vec{x})|$$
   We can write:
   $$\bm{f}(\vec{x}) = O(\bm{g}(\vec{x}))$$
\section{Differentiability}
   \subsection{Derivatives as linear functions}
   $\mathrm{d}\bm{f}|_{\vec{x}}: \mathbb{R}^n \to \mathbb{R}^m$ is a linear function: \\
   For scalar functions:
   $$\mathrm{d}f|_{\vec{t}} = \sum\limits_{i = 1}^n \frac{\partial f}{\partial x_i}\mathrm{d}_{x_i}, \mathbb{R}^n \to \mathbb{R}^m$$
   For vector functions: \\
   $\bm{f} = (f_1, ..., f_m), \Omega \subset \mathbb{R}^n \to \mathbb{R}^m$
   $$\mathrm{d}\bm{f}|_{\vec{t}} = (\mathrm{d}f_1|_{\vec{t}}, ...,\mathrm{d}f_m|_{\vec{t}})$$
   \subsection{Chain Rule}
   Given $\bm{f}: \Omega_1 \subset \mathbb{R}^n \to \mathbb{R}^m$ and $\bm{g}: \Omega_2 \subset \mathbb{R}^m \to \mathbb{R}^l$ such that: \\
   $\bm{f}(\Omega_1) \subset \Omega_2$, and $\bm{f}$ is differentiable at $\vec{t} \in \Omega_1$ and $\bm{g}$ is differentiable at $\bm{f}(\vec{t})\in \Omega_2$, then
   $$\mathrm{d}(\bm{g} \circ \bm{f})|_{\vec{t}} = (\mathrm{d}_{\bm{g}}|_{\bm{f}(\vec{t})}) \circ (\mathrm{d}\bm{f}|_{\vec{t}}): \mathbb{R}^n \to \mathbb{R}^l$$
   $$[\mathrm{d}(\bm{f} \circ \bm{g})|_{\vec{t}}] = [\mathrm{d}_{\bm{g}}|_{\bm{f}(\vec{t})}] \cdot [\mathrm{d}\bm{f}|_{\vec{t}}]$$
   \subsubsection{Notes}
   Continuous and partially differentiable $\nRightarrow$ differentiable \\
   \subsection{continuously differentiable functions}
   A function $\bm{f}: \Omega \subset \mathbb{R}^n \to \mathbb{R}^m$, where $\Omega$ is open, is called \textit{continuously differentiable} at $\vec{x} \in \Omega$ iff. \\
   $\bm{f}$ is partially differentiable at $\vec{x}$ and all of its partial Derivatives are continuous at $\vec{x}$. \\
   If $\bm{f}$ is continuously differentiable at every point in $\Omega$, we say that $\bm{f}$ is a \textit{continuously differentiable function}, and adopt the notation $\bm{f} \in C^1(\Omega; \mathbb{R}^n)$
\end{document}