%!TEX program = pdflatex
\documentclass[11pt, a4paper]{article}
\usepackage{amsfonts, setspace, mathptmx, bm, amsmath, color}
\renewcommand{\baselinestretch}{1.5}
\begin{document}

\title{Assignment 2}
\author{Pan Hao 2019012632}
\maketitle
\section*{Question 1}
\paragraph{i.}
$$\begin{aligned}
    \bm{f}(x, y) &= \bm{f}(1, 1) + [(x - 1) - (y - 1)] + \frac{1}{2!}[(x - 1) ^ 2 - 2(x - 1)(y - 1) + (y - 1) ^ 2] + \\&\frac{1}{3!}[(x - 1) ^ 3 - 2(x - 1) ^ 2(y - 1) - 2(x - 1) ^ 2(y - 1) + (y - 1) ^ 3] + o(||(x, y) - (1, 1)|| ^ 3) \\
    &= 1 + [x - y + \frac{1}{2}(x - y) ^ 2 + \frac{1}{6}(x - y) ^ 3] + o(||(x, y) - (1, 1)|| ^ 3)
\end{aligned}$$
\paragraph{ii.}
$$\begin{aligned}
    \bm{f}(x, y) &= 1 + [x - y + \frac{1}{2}(x - y) ^ 2] + R_k \\
    &= 1 + [x - y + \frac{1}{2}(x - y) ^ 2] + \frac{1}{6}(x - y) ^ 3 \bm{f}(\vec{\xi})
\end{aligned}$$
\paragraph{iii.} Take point$(1, \frac{1}{2})$ into Taylor expansion in \textbf{i}:
$$\begin{aligned}
    \bm{f}(1, \frac{1}{2}) &= 1 + [\frac{1}{2} + \frac{1}{2}(\frac{1}{2}) ^ 2 + \frac{1}{6}(\frac{1}{2}) ^ 3] + o(||(x, y) - (1, 1)|| ^ 3)\\
    \Rightarrow
    \sqrt{\mathrm{e}} &\approx \frac{79}{48}
\end{aligned}$$
\paragraph{iv.} Take point $(1, \frac{1}{2})$ into Taylor expansion in \textbf{ii}:
$$\begin{aligned}
    \bm{f}(x, y) &= 1 + [\frac{1}{2} + \frac{1}{2}(\frac{1}{2}) ^ 2] + \frac{1}{6}(\frac{1}{2}) ^ 3 \bm{f}(\vec{\xi}) \\
    \Leftrightarrow
    \sqrt{\mathrm{e}} &= \frac{13}{8} + \frac{1}{48}\bm{f}(\vec{\xi})
\end{aligned}$$
Since $\vec{\xi}$ is in the open interval between point $(1, 1)$ and $(1, \frac{1}{2})$, then $1 \leq \bm{f}(\vec{\xi}) \leq \sqrt{\mathrm{e}}$. Therefore, 
$$\begin{aligned}
    \sqrt{e} &\geq \frac{13}{8} + \frac{1}{48} = \frac{79}{48} \\
    \sqrt{e} &\leq \frac{13}{8} + \frac{1}{48}\sqrt{\mathrm{e}} \\
    &\leq \frac{78}{47} \\
    \Rightarrow
    \frac{79}{48} &\leq \sqrt{\mathrm{e}} \leq \frac{78}{47}
\end{aligned}$$

\section*{Question 2}
\paragraph{i.} \textbf{Proof:}
$$\nabla F = 
\begin{bmatrix}
    -2\mathrm{e}^{2z}x \\
    3y^2 \\
    8 - 2\mathrm{e}^{2z}x^2
\end{bmatrix}$$
Suppose that $\nabla F$ would vanish, then $x, y \equiv 0$. Consider $Z(F) = \{(x, y, z) | -x^2\mathrm{e}^{2z} + y^3 + 8z = 0\}$. For a point on $Z(F)$, if $x = y = 0$, then $z = 0$. However, $\nabla F$ at point $(0, 0, 0) = 8$ doesn't vanish. Therefore, $\nabla F$ never vanishes on $Z(F)$.
\paragraph{ii.}
Denote an arbitrary vector $\vec{x} = [x, y, z]^T$, $\vec{p} = [a, b, c]^T \in Z(F)$. At point $\vec{p}$, the normal vector of tangent plane of $Z(F)$ is $\nabla F|_{\vec{p}} = 
\begin{bmatrix}
    -2\mathrm{e}^{2c}a \\
    3b^2 \\
    8 - 2\mathrm{e}^{2c}a^2
\end{bmatrix}$. \\
Therefore, the point-normal form for the tangent plane is
$$(\vec{x} - \vec{p}) \cdot 
\begin{bmatrix}
    -2\mathrm{e}^{2c}a \\
    3b^2 \\
    8 - 2\mathrm{e}^{2c}a^2
\end{bmatrix} = 0$$
\paragraph{iii.}
Take two non-zero vectors which are orthonormal to $\nabla F|_{\vec{p}}$:
$$\vec{u} = 
\begin{bmatrix}
    -\frac{1}{2\mathrm{e}^{2c}a} \\
    \frac{1}{3b^2} \\
    0
\end{bmatrix}, \vec{v} = 
\begin{bmatrix}
    0 \\
    \frac{1}{3b^2} \\
    \frac{1}{8 - 2\mathrm{e}^{2c}a^2}
\end{bmatrix}$$
Then the parametric form for the tangent place is
$$\{\vec{x} = \vec{p} + s\vec{u} + t\vec{v} | s, t \in \mathbb{R}\}$$
\paragraph{iv.} 
\textbf{Proof:}
$$[\mathrm{d}\bm{F}] = 
\begin{bmatrix}
    -2\mathrm{e}^{2z}x & 3y^2 & 8 - 2\mathrm{e}^{2z}x^2
\end{bmatrix}$$
When $x \neq \pm 2\mathrm{e}^{-z}, 2\mathrm{e}^{2z}x^2 \neq 0$. According to the implicit function theorem, $\forall \vec{p} \in Z(F), \exists C^1$ map $\bm{\zeta}: B([a, b]^T, \delta) \to \mathbb{R}$ satisfying $F(x, y, \zeta(x, y)) = 0$.
$$\begin{aligned}
    \begin{bmatrix}
        \mathrm{d}\bm{\zeta}
    \end{bmatrix} &= -
    \begin{bmatrix}
        8 - 2\mathrm{e}^{2c}a^2
    \end{bmatrix}^{-1} \cdot 
    \begin{bmatrix}
        -2\mathrm{e}^{2c}a & 3b^2
    \end{bmatrix} \\
    &=
    \begin{bmatrix}
        \frac{\mathrm{e}^{2c}a}{4 - \mathrm{e}^{2c}a^2} & -\frac{3b^2}{8 - 2\mathrm{e}^{2c}a^2}
    \end{bmatrix}
\end{aligned}$$
\paragraph{v.}
\textbf{Proof:}
Consider $G(y, z, x) = -x^2\mathrm{e}^{2z} + y^3 + 8z$, then $F(x, y, z) = G(y, z, x)$
$$[\mathrm{d}\bm{G}] = 
\begin{bmatrix}
    3y^2 & 8 - 2\mathrm{e}^{2z}x^2 & -2\mathrm{e}^{2z}x
\end{bmatrix}$$
When $x \neq 0, -2\mathrm{e}^{2z}x \neq 0$. According to the implicit function theorem, $\forall \vec{p} \in Z(F), \exists C^1$ map $\bm{\xi}: B([b, c]^T, \delta) \to \mathbb{R}$ satisfying $G(z, y, \xi(y, z)) = F(\xi(y, z), y, z) = 0$.
$$\begin{aligned}
    \begin{bmatrix}
        \mathrm{d}\bm{\xi}
    \end{bmatrix} &= -
    \begin{bmatrix}
        -2\mathrm{e}^{2c}a
    \end{bmatrix}^{-1} \cdot 
    \begin{bmatrix}
        3b^2 & 8 - 2\mathrm{e}^{2c}a
    \end{bmatrix} \\
    &=
    \begin{bmatrix}
        \frac{3b^2}{2\mathrm{e}^{2c}a} & \frac{4 - \mathrm{e}^{2c}a^2}{\mathrm{e}^{2c}a}
    \end{bmatrix}
\end{aligned}$$
\paragraph{vi.}
\textbf{Proof:}
At point $[0, -2, 1]^T$, according to \textbf{iv.},
$$\begin{bmatrix}
    \mathrm{d}\bm{\zeta}|_{[0, -2, 1]^T}
\end{bmatrix} = 
\begin{bmatrix}
    0 & -\frac{3}{2}
\end{bmatrix}$$
Thus,
$$\begin{aligned}
    \begin{bmatrix}
        \mathrm{d}\bm{g}
    \end{bmatrix} &=
    \begin{bmatrix}
        1 + \partial_x\bm{\zeta} & \partial_y\bm{\zeta} \\
        0 & 1
    \end{bmatrix} \\
    &=
    \begin{bmatrix}
        1 & -\frac{3}{2} \\
        0 & 1
    \end{bmatrix}
\end{aligned}$$
Its columns are independent, so the matrix is invertible. According to the inverse function theorem, $\exists \bm{f}: B([1, -2]^T, \varepsilon) \to f(B([1, -2]^T, \varepsilon))$ which is inverse to $\bm{g}: f(B([1, -2]^T, \varepsilon)) \to B([1, -2]^T, \varepsilon)$.
$$\begin{aligned}
    \begin{bmatrix}
        \mathrm{d}\bm{f}
    \end{bmatrix} &=
    \begin{bmatrix}
        \mathrm{d}\bm{g}
    \end{bmatrix}^{-1} \\
    &=
    \begin{bmatrix}
        1 & -\frac{3}{2} \\
        0 & 1
    \end{bmatrix}^{-1} \\
    &=
    \begin{bmatrix}
        1 & \frac{3}{2} \\
        0 & 1
    \end{bmatrix}
\end{aligned}$$

\section{Question 3}
$$f(x, y, z) = (x + y - 1)^2 + (x - y + 2)^2 + 2z^2 + \frac{1}{6}
z^3$$
\paragraph{i.}
$$\begin{aligned}
    \begin{bmatrix}
        \mathrm{d}\bm{f}
    \end{bmatrix} &= 
    \begin{bmatrix}
        2(x + y - 1) + 2(x - y + 2) & 2(x + y - 1) - 2(x - y + 2) & 4z + \frac{1}{2}z^2
    \end{bmatrix} \\
    &=
    \begin{bmatrix}
        4x + 2 & 4y - 6 & 4z +\frac{1}{2}z^2
    \end{bmatrix}
\end{aligned}$$
In $B(\vec{0}, 10)$, let $[\mathrm{d}\bm{f}] = \vec{0}$, we get stationary points $\vec{u} = [-\frac{1}{2}, \frac{3}{2}, 0]^T$, $\vec{v} = [-\frac{1}{2}, \frac{3}{2}, -8]^T$.
\paragraph{ii.}
$$H_{\bm{f}} =
\begin{bmatrix}
    4 & 0 & 0 \\
    0 & 4 & 0 \\
    0 & 0 & 4 + z
\end{bmatrix}$$
Consider two stationary points $\vec{u}, \vec{v}$:
$$H_{\bm{f}}(\vec{u}) =
\begin{bmatrix}
    4 & 0 & 0 \\
    0 & 4 & 0 \\
    0 & 0 & 4
\end{bmatrix} = 4I$$
Thus the Hessian at this point is positive definite and therefore is a local minimum.
$$H_{\bm{f}}(\vec{v}) =
\begin{bmatrix}
    4 & 0 & 0 \\
    0 & 4 & 0 \\
    0 & 0 & -4
\end{bmatrix}$$
For this diagonal matrix, its eigenvalues are $\lambda_1 = \lambda_2 = 4 > 0, \lambda_3 = -4 < 0$. Thus the Hessian at this point is not positive
semi-definite and therefore is not a local extrema.
\paragraph{iii.}
Denote $\varphi(x, y, z) = x^2 + y^2 + z^2 - 100 = 0$. Consider:
$$\begin{aligned}
    F(x, y, z, \lambda) &= f(x, y, z) + \lambda\varphi(x, y, z) \\
    &= (x + y - 1)^2 + (x - y + 2)^2 + 2z^2 + \frac{1}{6}z^3 + \lambda(x^2 + y^2 + z^2 - 100) \\
    \Rightarrow \partial_x\bm{F} &= 4x + 2 + 2\lambda x = 0 \\
    \partial_y\bm{F} &= 4y - 6 + 2\lambda y = 0 \\
    \partial_z\bm{F} &= 4z + \frac{1}{2}z^2 + 2\lambda z = 0 \\
    \Rightarrow \lambda &= -\frac{2x + 1}{x} \\
    \lambda &= \frac{3 - 2y}{y} \\
    \lambda &= \frac{4z + \frac{1}{2}z^2}{2z} (z \neq 0)\\
    \Rightarrow
    &\left\{
        \begin{aligned}
            y &= -3x \\
            z &= \frac{4}{x}
        \end{aligned} 
    \right.
\end{aligned}$$
Take $y = -3x, z = \frac{4}{x}$ into $x^2 + y^2 + z^2 = 100$:
$$\begin{aligned}
    x^2 + (-3x)^2 + (\frac{4}{x})^2 &= 100 \\
    \Leftrightarrow 5x^4 -50x^2 + 8 &= 0 \\
    \Rightarrow x^2_1 = \frac{25 + 3\sqrt{65}}{5}, x^2_2 &= \frac{25 - 3\sqrt{65}}{5} \\
    \Rightarrow x_1 = \sqrt{\frac{25 + 3\sqrt{65}}{5}}, x_2 &= -\sqrt{\frac{25 + 3\sqrt{65}}{5}} \\
    x_3 = \sqrt{\frac{25 - 3\sqrt{65}}{5}}, x_4 &= -\sqrt{\frac{25 - 3\sqrt{65}}{5}}
\end{aligned}$$
if $z = 0$, take $y = -3x$ into $x^2 + y^2 + z^2 = 100$:
$$\begin{aligned}
    10x^2 &= 100 \\
    \Rightarrow x_5 = \sqrt{10}, x_6 &= -\sqrt{10}
\end{aligned}$$
Since $\bm{f}$ is continuous on the boundary of $B(\vec{0}, 10)$, the extrema necessarily exist.
On the boundary: 
$$\begin{aligned}
    For\text{ } x_1: \bm{f} &= 268.07\\
    For\text{ } x_2: \bm{f} &= 141.93\\
    For\text{ } x_3: \bm{f} &= 375.68\\
    For\text{ } x_4: \bm{f} &= 34.32\\
    For\text{ } x_5: \bm{f} &= 268.25\\
    For\text{ } x_6: \bm{f} &= 141.75\\
\end{aligned}$$
Stationary point:
$$\bm{f} = 0$$
Therefore, Maxima = 375.68. minima = 0. 
\paragraph{iv.}
$$\bm{g}: \mathbb{R} \times (0, +\infty) \times [0, \frac{\pi}{2}) \to \mathbb{R}$$
$$g(u, v, w) = u^6 + u^3 + (\log v)^2 - 3\log v + \tan^2 w + \frac{1}{12}\tan^3 w$$
Obviously, $\bm{g}$ goes to infinity when $u, v, w$ go to their boundaries. Thus, the minima would only occur at the stationary point.
$$\begin{aligned}
    \begin{bmatrix}
        \mathrm{d}\bm{g}
    \end{bmatrix} &=
    \begin{bmatrix}
        6u^5 + 3u^2 & \frac{2\log v - 3}{v} & 2\sec^2 w\tan w + \frac{1}{4}\sec^2 w\tan^2 w
    \end{bmatrix} \\
    &\text{Let } [\mathrm{d}\bm{g}] = 0 \\
    &\Rightarrow 3u^2(2u^3 + 1) = 0 \\
    &\frac{2\log v - 3}{v} = 0 \\
    &2\sec^2 w\tan w + \frac{1}{4}\sec^2 w\tan^2 w = 0 \\
\end{aligned}$$
$$\begin{aligned}
    \Rightarrow u_1 = 0, u_2 &= (-\frac{1}{2})^{\frac{1}{3}} \\
    v &= \mathrm{e}^{\frac{3}{2}} \\
    w &= 0 \\
    g(0, \mathrm{e}^{\frac{3}{2}}, 0) &= -\frac{9}{4} \\
    g((-\frac{1}{2})^{\frac{1}{3}}, \mathrm{e}^{\frac{3}{2}}, 0) &= -\frac{9}{4} \\
    \Rightarrow g(0, \mathrm{e}^{\frac{3}{2}}, 0) = g(0, \mathrm{e}^{\frac{3}{2}}, 0) &= -\frac{9}{4}
\end{aligned}$$
Therefore, there are two minima for this function $[0, \mathrm{e}^{\frac{3}{2}}, 0]^T, [(-\frac{1}{2})^{\frac{1}{3}}, \mathrm{e}^{\frac{3}{2}}, 0]^T$
\end{document}