%!TEX program = pdflatex
\documentclass[11pt, a4paper]{article}
\usepackage{amsfonts, setspace, mathptmx, bm, amsmath, color}
\renewcommand{\baselinestretch}{1.5}
\begin{document}

\title{Assignment 1}
\author{Pan Hao}
\date{}
\maketitle
\section*{Question 1} 

\paragraph{i.}
$S = \{x \in \mathbb{R} | x = \sqrt {2} + k, k \in \mathbb{N} \}$ 

\paragraph{ii.}
\textbf{Proof:} For any arbitrary $k\in \mathbb{N}$, we define $S_k := (\sqrt{2} + k, \sqrt{2} + k + 1)$. \\ 
According to the definition, $S_k \subset \mathbb{R} \setminus S$. For arbitrary $x \in S_k$, let $\delta = min\{x- \sqrt{2} - k, \sqrt{2} + k + 1 - x\}$, $\exists B(x, \delta)$ is an open ball, thus $S_k$ is an open set.
Then the complement of $S$ in $\mathbb{R}$ $\overline{S} = \bigcup_{k \in \mathbb{N}} S_k$ is an open set. Therefore, $S$ is an closed set.

\paragraph{iii.}
\textbf{Proof:} Since $\mathbb{N}$ is unbounded (according to the Archimedes character), $S$ is an unbounded set.

\paragraph{iv.}
$B((0, 0, 4), 1)$

\paragraph{v.}
\textbf{Proof:} Obviously it is an open set since it is an oepn ball.

{\color{red} \paragraph{vi.}
\textbf{Proof:} $\forall \vec{x_1}, \vec{x_2} \in B(0, 0, 4), d(\vec{x_i}, (0, 0, 4)) < 1$. Therefore, according to triangular inequality, $d(\vec{x_1}, \vec{x_2}) \leq d(\vec{x_1}, (0, 0, 4)) + d(\vec{x_2}, (0, 0, 4)) < 2$. Thus, this open ball is bounded.}

\paragraph{vii.}
$diam(B) = 2$.\\
\textbf{Proof:} For arbitrary points $\vec{p}, \vec{q} \in B(\vec{x}, 1), d(\vec{x}, \vec{p}) < 1, d(\vec{x}, \vec{q}) < 1$, thus $d(\vec{x}, \vec{y}) < 2$ according to the triangular inequality.
Now consider two point sequences:
$$A_k = \{ (0, 0, 3 + \frac{1}{2 ^ k}) | k \in \mathbb{N} \}, B_k = \{ (0, 0, 5 - \frac{1}{2 ^ k}) | k \in \mathbb{N} \}$$
Obviously $A_k, B_k \subset B(\vec{x}, 1)$.
Then we can tell
$$\lim_{k \to \infty} d(A_k, B_k) = 2$$
Which means $\forall \varepsilon > 0, \exists N > 0$, when $k > N, d(A_k, B_k) > 2 - \varepsilon$. Then we can tell $sup(\vec{p}, \vec{q}) = 2$, which means $diam(B) = 2$.


\section*{Question 2}

\paragraph{i.}
$$\lim_{(x, y) \to (3.5)} (\frac{\sin (y - x)}{y - x}, \sqrt{y ^ 2 - x ^ 2})$$ exists \\
\textbf{Proof:} Both $\frac{\sin (y - x)}{y - x}$ and $\sqrt{y ^ 2 - x ^ 2}$ are continuous at point $(3, 5)$, thus the limit is equal to $(\frac{\sin 2}{2}, 4)$.

\paragraph{ii.}
$$\lim_{(x, y) \to (0, 0)} \frac{x ^ 4 - y ^ 3}{x ^ 3 - y ^ 4}$$ doesn't exist. \\
\textbf{Proof:} Replace $y$ with $kx (k \in \mathbb{R})$, then the original formula is equal to 
$$\begin{aligned}
    \lim_{x \to 0} \frac{x ^ 4 - k ^ 3 x ^ 3}{x ^ 3 - k ^ 4 x ^ 4} &= \lim_{x \to 0} \frac{x - k ^ 3}{1 - k ^ 4 x} \\
    &= -k ^ 3
\end{aligned}$$ which is uncertain. Therefore, the limit doesn't exist.

\paragraph{iii.}
$$\lim\limits_{^{x \to +\infty}_{y \to +\infty}} \frac{x ^ 3 - y ^ 3}{x ^ 4 - y ^ 4}$$ exists. \\
\textbf{Proof:} The original formula is equal to 
$$\lim\limits_{^{x \to +\infty}_{y \to +\infty}} \frac{x ^ 2 + xy + y ^ 2}{x ^ 3 + x ^ 2 y + x y ^ 2 + y ^ 3}$$ which is obviously no less than zero. Then
$$\begin{aligned}
\lim\limits_{^{x \to +\infty}_{y \to +\infty}} \frac{x ^ 2 + xy + y ^ 2}{x ^ 3 + x ^ 2 y + x y ^ 2 + y ^ 3} &= 
\lim\limits_{^{x \to +\infty}_{y \to +\infty}} \frac{x ^ 2 + xy + y ^ 2 + \frac{y ^ 3}{x} - \frac{y ^3}{x}}{x ^ 3 + x ^ 2 y + x y ^ 2 + y ^ 3} \\
&= 
\lim\limits_{^{x \to +\infty}_{y \to +\infty}} (\frac{1}{x} - \frac{y ^ 3}{x ^ 4 + x ^ 3 y + x ^ 2 y ^ 2 + x y ^ 3}) \\
&\leq 
\lim_{x \to +\infty} \frac{1}{x} = 0
\end{aligned}$$
As a result, 
$$0 \leq \lim\limits_{^{x \to +\infty}_{y \to +\infty}} \frac{x ^ 3 - y ^ 3}{x ^ 4 - y ^ 4} \leq 0$$
Therefore, the limit is equal to 0.

\paragraph{iv.}

$$\lim_{x \to +\infty}\lim_{y \to -\infty} \frac{x ^ 3 - y ^ 3}{x ^ 4 - y ^ 4}$$ exists. \\
\textbf{Proof:} Replace $y$ with $u = \frac{1}{y}$, then
$$\lim_{y \to -\infty} \frac{x ^ 3 - y ^ 3}{x ^ 4 - y ^ 4} = \lim_{u \to 0 ^ -} \frac{x ^ 3 - \frac{1}{u ^ 3}}{{x ^ 4 - \frac{1}{u ^ 4}}} = \lim_{u \to 0 ^ -} \frac{(u ^ 3 x ^ 3 - 1)u}{u ^ 4 x ^ 4 - 1} = 0$$
Therefore,
$$\lim_{x \to +\infty}\lim_{y \to -\infty} \frac{x ^ 3 - y ^ 3}{x ^ 4 - y ^ 4} = \lim_{x \to +\infty} 0 = 0$$

\paragraph{v.}
$$\lim_{(x, y) \to (\mathrm{e}, 0)}(1 + 2020y)^{\frac{1}{y - x ^ 2 y ^ 2}}$$ exists. \\
\textbf{Proof:} Consider 
$$\bm{f}: \mathbb{R} ^ 2 \to \mathbb{R} ^ 2, \bm{f}(x, y) = ((1 + 2020y) ^ \frac{1}{2020y}, \frac{2020y}{y - x ^ 2 y ^ 2})$$ = 
$$\bm{f}: \mathbb{R} ^ 2 \to \mathbb{R} ^ 2, \bm{f}(x, y) = ((1 + 2020y) ^ \frac{1}{2020y}, \frac{2020}{1 - x ^ 2 y})$$,
Therefore, 
$$\lim_{(x, y) \to (\mathrm{e}, 0)} \bm{f}(x, y) = (e, 2020)$$
$$\bm{g}: \mathbb{R} ^ 2 \to \mathbb{R}: \bm{g}(u, v) = u ^ v$$
Then invoke the composition rule:
$$\lim_{(u, v) \to (\mathrm{e}, 2020)} \bm{g}(u, v) = \mathrm{e} ^ {2020}$$

\paragraph{vi.}
$$\lim_{(x, y) \to (3, +\infty)} \frac{\log(x + y)}{x ^ 2 + y ^ 2}$$ exists. \\
\textbf{Proof:} According to the fundemental inequality, $x ^ 2 + y ^ 2 \geq \frac{x ^ 2 + y ^ 2}{2}$, then
$$\frac{\log(x + y)}{x ^ 2 + y ^ 2} \leq \frac{\log(x + y)}{\frac{(x + y) ^ 2}{2}}$$
Replace $x ^ 2 + y ^ 2$ with $u$. When $(x ^ 2 + y ^ 2) \to (3, \infty), u \to \infty$. Then 
$$\lim_{(x, y) \to (3, \infty)} \frac{\log(x + y)}{\frac{(x + y) ^ 2}{2}} = \lim_{u \to \infty} \frac{2\log u}{u ^ 2} = 0$$
Meanwhile, the original formula is no less than zero, which means
$$0 \leq \lim_{(x, y) \to (3, +\infty)} \frac{\log(x + y)}{x ^ 2 + y ^ 2} \leq 0$$
Therefore, the limit is equal to 0.

\section*{Question 3}

\paragraph{i.}
\textbf{Proof:} Denote $\vec{h}$ as $\begin{bmatrix}
    h_x \\
    h_y
\end{bmatrix}$, $\vec{t}$ as
$\begin{bmatrix}
    x \\
    y
\end{bmatrix}$. 
Take the linear function $\bm{L}(\vec{h})$ as $\begin{bmatrix}
    2 & 0 \\
    0 & 2 \\
    2 & 1
\end{bmatrix}
\begin{bmatrix}
    h_x \\
    h_y \\
\end{bmatrix}$. \\ 
Consider
$$\begin{aligned}
    \bm{f}(\vec{t} + \vec{h}) - \bm{f}(\vec{t}) - \bm{L}(\vec{h}) &=
    \begin{bmatrix}
       2(x + h_x) + 3 - (2x + 3) - 2h_x \\
       (x + h_x) ^ 2 + (y + h_y) ^ 2 - (x ^ 2 + y ^ 2) - 2h_y \\
       (y + h_y)\mathrm{e} ^ {2(x + h_x)} - y\mathrm{e} ^ {2x} - (2h_x + h_y)
    \end{bmatrix} \\
    &=
    \begin{bmatrix}
        0 \\
        h_x ^ 2 + 2xh_x + 2h_y(y - 1) + h_y ^ 2 \\
        y\mathrm{e} ^ {2x} (\mathrm{e} ^ {2h_x} - 1) + h_y(\mathrm{e} ^ {2(x + h_x)} - 1) - 2h_x
    \end{bmatrix}
\end{aligned}$$
When $\vec{t} = (0, 1)$,
$$\bm{f}(\vec{t} + \vec{h}) - \bm{f}(\vec{t}) - \bm{L}(\vec{h}) = 
\begin{bmatrix}
    0 \\
    h_x ^ 2 + 2xh_x + 2h_y(y - 1) + h_y ^ 2 \\
    \mathrm{e} ^ {2h_x} - 1 + h_y(\mathrm{e} ^ {2h_x} - 1) - 2h_x
\end{bmatrix}$$
$$\lim_{\vec{h} \to 0} \frac{||\bm{f}(\vec{t} + \vec{h}) - \bm{f}(\vec{t}) - \bm{L}(\vec{h})||}{|| \vec{h} ||}$$
=
$$\lim_{\vec{h} \to 0} \sqrt{\frac{(h_x ^ 2 + 2xh_x + 2h_y(y - 1) + h_y ^ 2) ^ 2 + (\mathrm{e} ^ {2h_x} - 1 + h_y(\mathrm{e} ^ {2h_x} - 1) - 2h_x) ^ 2}{h_x ^ 2 + h_y ^ 2}}$$
=
$$\lim_{\vec{h} \to 0} \sqrt{\frac{(h_x ^ 2 + h_y ^ 2) ^ 2 + (2h_x h_y) ^ 2}{h_x ^ 2 + h_y ^ 2}} {\color{red} (\vec{h} \to 0, 2h_x = \mathrm{e} ^ {2h_x} - 1 + o(2h_x))}$$
=
$$\lim_{\vec{h} \to 0} \sqrt{h_x ^ 2 + h_y ^ 2 + \frac{4}{\frac{1}{h_y ^ 2} + \frac{1}{h_x ^ 2}}} = 0$$
Therefore, $\bm{f}$ is differentiable at $(0, 1)$.

\paragraph{ii.}
At $(x, y) = (0, 1)$
$$\begin{aligned}
\frac{\partial{f_1}}{\partial{x}} &= 2,
\frac{\partial{f_1}}{\partial{y}} = 0 \\
\frac{\partial{f_2}}{\partial{x}} &= 2x = 0,
\frac{\partial{f_2}}{\partial{y}} = 2y = 2 \\
\frac{\partial{f_3}}{\partial{x}} &= 2y\mathrm{e} ^ {2x} = 2,\frac{\partial{f_2}}{\partial{y}} = \mathrm{e} ^ {2x} = 1
\end{aligned}$$

\paragraph{iii.}
$$[\mathrm{d}\bm{f}|_{(0, 1)}] = 
\begin{bmatrix}
    2 & 0 \\
    0 & 2 \\
    2 & 1
\end{bmatrix}$$
\end{document}