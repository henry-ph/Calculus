%!TEX program = pdflatex
\documentclass[11pt, a4paper]{article}
\usepackage{amsfonts, setspace, mathptmx, bm, amsmath, color, amssymb, amsthm, mathrsfs, txfonts}
\renewcommand{\baselinestretch}{1.5}
\begin{document}
\title{Assignment 5}
\author{Pan Hao}
\date{}
\maketitle

\section*{Question 1}
\paragraph{i.}
$$\sum\limits_{n = 1}^\infty 3^{-2n + \frac{1}{n}}$$ is converged. \\
\textbf{Proof:}
$$\begin{aligned}
    \lim\limits_{n \to \infty} (3^{-2n + \frac{1}{n}})^\frac{1}{n} &= \lim\limits_{n \to \infty} 3^{-2 + \frac{1}{n^2}} \\
    &= \frac{1}{9} < 1
\end{aligned}$$
According to the root test, $\sum\limits_{n = 1}^\infty 3^{-2n + \frac{1}{n}}$ is converged.

\paragraph{ii.}
$$\sum\limits_{n = 2020}^\infty \sin(\pi(n^2 - n + \frac{2}{n}))$$
is diverged. \\
\textbf{Proof:}
$$\begin{aligned}
    \sin(\pi(n^2 - n + \frac{2}{n})) &= \sin(n(n - 1)\pi + \frac{2}{n}\pi) \\
    &= \sin(\frac{2\pi}{n}) \\
    \lim\limits_{n \to \infty} \frac{\sin(\frac{2\pi}{n})}{\frac{1}{n}} &= \lim\limits_{n \to \infty}\frac{\frac{2\pi}{n} + o(\frac{1}{n})}{\frac{1}{n}} \\
    &= 2\pi + o(1)
\end{aligned}$$
Since $\sum\limits_{n=2020}^\infty \frac{1}{n}$ is diverged, $\sum\limits_{n = 2020}^\infty \sin(\pi(n^2 - n + \frac{2}{n}))$ is diverged as well.

\paragraph{iii.}
$$\sum\limits_{n=2020}^\infty \frac{\tanh(n)}{n - \cos(n)}$$
is diverged. \\
\textbf{Proof:}
$$\begin{aligned}
    \lim\limits_{n \to \infty} \frac{\frac{\tanh(n)}{n - \cos(n)}}{\frac{1}{n}} &= \lim\limits_{n \to \infty} \frac{\tanh(n)}{1 - \frac{\cos(n)}{n}} \\
    &= 1
\end{aligned}$$
Since $\sum\limits_{n=2020}^\infty \frac{1}{n}$ is diverged, $\sum\limits_{n=2020}^\infty \frac{\tanh(n)}{n - \cos(n)}$ is diverged as well.

\paragraph{iv.}
$$\sum\limits_{n = 1}^\infty (1 - \frac{1}{n})^{n^2}$$
is converged. \\
\textbf{Proof:}
$$\begin{aligned}
    ((1 - \frac{1}{n})^{n^2})^{\frac{1}{n}} &= (1 - \frac{1}{n})^n \\
    \lim\limits_{n \to \infty} (1 - \frac{1}{n})^n &= \lim\limits_{n \to \infty} ((1 + (-\frac{1}{n}))^{-n})^{-1} \\
    &= \frac{1}{\mathrm{e}} < 1
\end{aligned}$$
According to the root test, $\sum\limits_{n = 1}^\infty (1 - \frac{1}{n})^{n^2}$ is converged.

\paragraph{v.}
$$\sum\limits_{n=0}^\infty \frac{n!}{n^n}$$
is converged. \\
\textbf{Proof:}
$$\begin{aligned}
    \frac{\frac{(n + 1)!}{(n+1)^{n+1}}}{\frac{n!}{n^n}} &= \frac{(n + 1)n^n}{(n+1)^{n+1}} \\
    &= (\frac{n}{n+1})^n < 1
\end{aligned}$$
According to the ratio test, $\sum\limits_{n=0}^\infty \frac{n!}{n^n}$ is converged.

\paragraph{vi.} 
$$\sum\limits_{n=1}^\infty \frac{n\cos(n)}{n^2+1}$$
is converged. \\
\textbf{Proof:} First consider $\sum\limits_{n=1}^\infty \cos(n)$:
$$\begin{aligned}
    2\cos(1)\sin(\frac{1}{2}) &= \sin(\frac{3}{2}) - \sin(\frac{1}{2}) \\
    2\cos(2)\sin(\frac{1}{2}) &= \sin(\frac{5}{2}) - \sin(\frac{3}{2}) \\
    &\cdots \\
    2\cos(n)\sin(\frac{1}{2}) &= \sin(n + \frac{1}{2}) - \sin(n - \frac{1}{2}) \\
    \Rightarrow \sum\limits_{n=1}^\infty \cos(n) &= \frac{\sin(n+\frac{1}{2}) - \sin(\frac{1}{2})}{2\sin(\frac{1}{2})} \\
    &< \frac{1 + 1}{2\sin(\frac{1}{2})} = \csc(\frac{1}{2})
\end{aligned}$$
Thus $\sum\limits_{n=1}^\infty \cos(n)$ is bounded. Then consider sequence $\{\frac{n}{n^2+1}\} (n \geq 1)$ which is monotonic.
$$\begin{aligned}
    \lim\limits_{n \to \infty} \frac{n}{n^2+1} &= \lim\limits_{n \to \infty} \frac{1}{n + \frac{1}{n}} \\
    &= 0
\end{aligned}$$
Thus $\{\frac{n}{n^2+1}\} (n \geq 1)$ monotonically goes to zero. Therefore, according to Dirichlet test, $\sum\limits_{n=1}^\infty \frac{n\cos(n)}{n^2+1}$ is converged.

\paragraph{vii.}
$$\sum\limits_{n=2020}^\infty \frac{\log(\log(n))}{n(\log(n))^2}$$
is converged. \\
\textbf{Proof:} Since $\frac{\log(\log(n))}{n(\log(n))^2}$ is positive and monotonically decreasing, according to the integral comparison test, $\sum\limits_{n=2020}^\infty \frac{\log(\log(n))}{n(\log(n))^2}$ agrees in convergent/divergent with $\int\limits_{n=2020}^\infty \frac{\log(\log(n))}{n(\log(n))^2} \mathrm{d}n$.
$$\begin{aligned}
    \int\limits_{n=2020}^\infty \frac{\log(\log(n))}{n(\log(n))^2} \mathrm{d}n &= \int\limits_{n=2020}^\infty \frac{\log(\log(n))}{(\log(n))^2} \mathrm{d}(\log(n)) \\
    &= \int\limits_{u=\log(2020)}^\infty \frac{\log(u)}{u^2}\mathrm{d}u \\
    &= \int\limits_{u=\log(2020)}^\infty \frac{\log(u)}{u^{\frac{1}{2}}} \cdot \frac{1}{u^{\frac{3}{2}}}\mathrm{d}u
\end{aligned}$$
First consider $\frac{\log(u)}{u^{\frac{1}{2}}}$:
$$\begin{aligned}
    \frac{\mathrm{d}}{\mathrm{d}u} \frac{\log(u)}{u^{\frac{1}{2}}} &= \frac{u^{-\frac{1}{2}} - \frac{1}{2}\log(u)u^{-\frac{1}{2}}}{u} \\
    &= (1 - \frac{1}{2}\log(u))u^{-\frac{3}{2}} < 0 \\
    \lim\limits_{n \to \infty} \frac{\log(u)}{u^{\frac{1}{2}}} &= \lim\limits_{n \to \infty} \frac{\frac{1}{u}}{\frac{1}{2}u^{-\frac{1}{2}}} = 0
\end{aligned}$$
Thus $\frac{\log(u)}{u^{\frac{1}{2}}}$ monotonically bounded. Since $\int\limits_{u=\log(2020)}^\infty \frac{1}{u^{\frac{3}{2}}}\mathrm{d}u$ converges, according to the Abel test, $\int\limits_{n=2020}^\infty \frac{\log(\log(n))}{n(\log(n))^2} \mathrm{d}n$ is converged. Therefore, $\sum\limits_{n=2020}^\infty \frac{\log(\log(n))}{n(\log(n))^2}$ is converged as well.


\section*{Question 2}

\paragraph{i.}
$$\sum\limits_{n=2}^\infty \frac{n^x}{(\log(n))^n}$$
Consider the ratio test:
$$\begin{aligned}
    \frac{\frac{(n+1)^x}{(\log(n+1))^{n+1}}}{\frac{n^x}{(\log(n))^n}} &= (\frac{n+1}{n})^x \cdot \frac{(\log(n))^n}{(\log(n+1))^{n+1}} \\
    \frac{(\log(n))^n}{(\log(n+1))^{n+1}} &= \exp(n\log(\log(n)) - (n+1)\log(\log(n+1))) \\
    &= \exp(n\log\frac{\log(n)}{\log(n+1)} - \log(\log(n+1))) \\
    \text{Since } \log(n) < \log(n+1) &\text{, } \frac{\log(n)}{\log(n+1)} < 0\\
    \lim\limits_{n \to \infty} n\log\frac{\log(n)}{\log(n+1)} - \log(\log(n+1)) &= -\infty \\
    \lim\limits_{n \to \infty} \exp(n\log\frac{\log(n)}{\log(n+1)} - \log(\log(n+1))) &= 0
\end{aligned}$$
Since $(\frac{n+1}{n})^x$ goes to $1$ when $n$ goes to $\infty$, $(\frac{n+1}{n})^x \cdot \frac{(\log(n))^n}{(\log(n+1))^{n+1}}$ goes to $0$. Therefore, $\sum\limits_{n=2}^\infty \frac{n^x}{(\log(n))^n}$ is converged for all $x \in \mathbb{R}$.

\paragraph{ii.}
$$\sum\limits_{n=1}^\infty x^{-1 - \frac{1}{2} - \cdots - \frac{1}{n}}$$
According to the logarithmic test:
$$\begin{aligned}
    \log(\frac{1}{a_n}) &= \log(x^{1 + \frac{1}{2} + \cdots + \frac{1}{n}}) \\
    &= (1 + \frac{1}{2} + \cdots + \frac{1}{n})\log(x) \\
    &= (\log(n) + \gamma)\log(x) \\
    L_n &= \frac{\log(\frac{1}{a_n})}{\log(n)} \\
    &= \frac{(\log(n) + \gamma)\log(x)}{\log(n)} \\
    &= \log(x) + \frac{\gamma}{\log(n)}\log(x) > \log(x)
\end{aligned}$$ 
Therefore, if $\sum\limits_{n=1}^\infty x^{-1 - \frac{1}{2} - \cdots - \frac{1}{n}}$ converges, $\log(x) \geq 1$. Thus the domain of convergence is $x > \mathrm{e}$.
\end{document}