%!TEX program = pdflatex
\documentclass[11pt, a4paper]{article}
\usepackage{amsfonts, setspace, mathptmx, bm, amsmath, color, amssymb, amsthm, mathrsfs}
\renewcommand{\baselinestretch}{1.5}
\begin{document}
\title{Answer Sheet}
\author{Pan Hao, 2019012632}
\date{}
\maketitle
\section*{Part I}
\paragraph{1-5} F T F T T
\paragraph{6-10} T T T T F

\section*{Part II}
\paragraph{11.} 6
\paragraph{12.} $\mathrm{e}^{-1}$
\paragraph{13.}
$\begin{bmatrix}
    24 & 0 \\
    8 & 4
\end{bmatrix}$
\paragraph{14.}
$\begin{bmatrix}
    -\cos(3)\mathrm{e}^{-\sin 3} - 6\sin 10 \\
    3\cos(3)\mathrm{e}^{-\sin 3} + 2\sin 10
\end{bmatrix}$
\paragraph{15.}
$(\begin{bmatrix}
    x \\
    y \\
\end{bmatrix} -
\begin{bmatrix}
    -\sqrt[3]{4} \\
    \sqrt[3]{2}
\end{bmatrix}) \cdot
\begin{bmatrix}
    1 \\
    2\sqrt[3]{2}
\end{bmatrix} = 0$
\paragraph{16.} $1 + y + y^2$
\paragraph{17.} $(0, 0), (-\frac{1}{3}, -1)$
\paragraph{18.} $(0, 0)$: neither; $(-\frac{1}{3}, -1)$: local maximum
\paragraph{19.} $\frac{\mathrm{d}y}{\mathrm{d}x} = -12, \frac{\mathrm{d}z}{\mathrm{d}x} = 4$
\paragraph{20.} $\frac{17}{12}$

\section*{Part III}
\paragraph{21.}
$$\lim\limits_{\vec{x} \to [3, 1, -4]^T} \frac{1 - xy}{x^2 + y^2 + z^2 + 3}$$
\textbf{Proof:} The limit exists. This is a well-defined elementary function around $(x, y, z) = (3, 1, -4)$. The limit exists and is equal to $-\frac{2}{29}$. \\
$$\Omega_1 := \{[x, y]^T \in \mathbb{R}|x \neq 0\}, \lim\limits_{\Omega_1 \ni \vec{x} \to [0, 1]^T} (y + x) ^ \frac{1}{x}$$ 
\textbf{Proof:} The limit exists. First observe that we know the following limits exist:
$$\lim\limits_{x \to 0} (1 + x) ^\frac{1}{x} = \mathrm{e}$$
Therefore,
$$\lim\limits_{\Omega_1 \ni \vec{x} \to [0, 1]^T} (y + x) ^ \frac{1}{x} = \lim\limits_{x \to 0} (1 + x) ^\frac{1}{x} = \mathrm{e}$$ \\
$$\Omega_2 := \mathbb{R}^2\setminus \{\vec{0}\}, \lim\limits_{\Omega_2 \ni \vec{x} \to \vec{0}} \frac{\tan(xy)\log(1 + |xy|)}{x^2 + y^2}$$
\textbf{Proof:} The limit exists. First we observe that
$$\lim\limits_{\Omega_2 \ni \vec{x} \to \vec{0}} \frac{\tan(xy)\log(1 + |xy|)}{x^2 + y^2} \geq 0$$
Since $x^2 + y^2 \geq 2xy$, we observe that
$$\begin{aligned}
    \lim\limits_{\Omega_2 \ni \vec{x} \to \vec{0}} \frac{\tan(xy)\log(1 + |xy|)}{x^2 + y^2} &\leq \lim\limits_{\Omega_2 \ni \vec{x} \to \vec{0}} \frac{\tan(xy)\log(1 + |xy|)}{2xy} \\ 
    &= \lim\limits_{\Omega_2 \ni \vec{x} \to \vec{0}} \frac{(xy)^2 + o((xy)^2)}{2xy} \\
    &= 0 
\end{aligned}$$
According to the sandwich theorem, the limits is equal to 0.

\paragraph{22.}
$$\begin{bmatrix}
        \mathrm{d}H
\end{bmatrix} =
\begin{bmatrix}
        2x & 6y & \sin(2z)
\end{bmatrix}$$
$$\bm{H} = 
\begin{bmatrix}
    2 & 0 & 0 \\
    0 & 6 & 0 \\
    0 & 0 & 2\cos(2z)   
\end{bmatrix}$$
Let $[\mathrm{d}H] = \vec{0}$, we get stationary points $\vec{p} = [0, 0, \frac{k\pi}{2}]^T$ \\
First consider $k = 2n, n \in \mathbb{N}^*$:
$$\bm{H}_{\vec{p}} = 
\begin{bmatrix}
    2 & 0 & 0 \\
    0 & 6 & 0 \\
    0 & 0 & 1
\end{bmatrix}$$
Since this is a diagonal matrix, its eigenvalues are $\lambda_1 = 6> \lambda_2 = 2 > \lambda_3 = 1> 0$.
Therefore, Hessian Matrixes at $\vec{p}$ are positive definite, $\vec{p}$ are local minima. \\
Then consider $k = 2n + 1, n \in \mathbb{N}^*$:
$$\bm{H}_{\vec{p}} = 
\begin{bmatrix}
    2 & 0 & 0 \\
    0 & 6 & 0 \\
    0 & 0 & -1
\end{bmatrix}$$
Since this is a diagonal matrix, its eigenvalues are $\lambda_1 = 6> \lambda_2 = 2 > 0, \lambda_3 = -1 < 0$. Therefore, the Hessian Matrix are neither positive definite nor semi-positive definite, these stationary points are not extrema. \\
Global minima of $H$ is $min(\sin(\frac{k\pi}{2})) = 0 (k = 2n, n \in \mathbb{N}^*)$

\paragraph{24.}
$$\begin{aligned}
    \int_{0}^{\infty}\int_{1}^{7}\frac{x^2}{x^6 + y^2}\mathrm{d}y\mathrm{d}x &= \int_{1}^{7}\int_{0}^{\infty}\frac{x^2}{x^6 + y^2}\mathrm{d}x\mathrm{d}y \\
    \frac{x^2}{x^6 + y^2} &= \frac{1}{x^4 + \frac{y^2}{x^2}} < \frac{1}{x^4}
\end{aligned}$$
Since $\int_{1}^{\infty}\frac{1}{x^4}\mathrm{d}x$ is convergent, $\int_{1}^{\infty}\frac{x^2}{x^6 + y^2}$ is also convergent. Therefore,
$$\begin{aligned}
    \int_{0}^{\infty}\frac{x^2}{x^6 + y^2}\mathrm{d}x &= \frac{1}{3}\int_{0}^{\infty}\frac{1}{x^6 + y^2}\mathrm{d}(x^3) \\
    &= \frac{1}{3y^2}\int_{0}^{\infty}\frac{1}{(\frac{x^3}{y})^2 + 1}\mathrm{d}(x^3) \\
    &= \frac{1}{3y}\int_{0}^{\infty}\frac{1}{(\frac{x^3}{y})^2 + 1}\mathrm{d}(\frac{x^3}{y}) \\
    &= \frac{1}{3y}\arctan(\frac{x^3}{y})\Big|_{0}^{\infty} = \frac{\frac{\pi}{2}}{3y} \\
    \int_{1}^{7}\int_{0}^{\infty}\frac{x^2}{x^6 + y^2}\mathrm{d}x\mathrm{d}y &= \int_{1}^{7} \frac{\frac{\pi}{2}}{3y} \mathrm{d}y \\
    &= \frac{\pi}{6}\log y\Big|_{1}^{7} \\
    &= \frac{\pi}{6}\log 7
\end{aligned}$$
\paragraph{25.}
$$\begin{aligned}
    M &= \int_{0}^{1}\int_{0}^{1}\int_{0}^{1} \exp(x - y + 2z) \mathrm{d}x\mathrm{d}y\mathrm{d}z \\
    &=\int_{0}^{1}\int_{0}^{1}(\exp(1-y+2z) - \exp(-y+2z))\mathrm{d}y\mathrm{d}z \\
    &=\int_{0}^{1}(\exp(1+2z)+\exp(2z-1)+\exp(2z))\mathrm{d}z \\
    &= \frac{1}{2}\mathrm{e}^3 - \mathrm{e}^2 - \frac{1}{2}\mathrm{e}^{-1} + 1 \\
    \hat{x} &= \frac{1}{M} \int_{0}^{1}\int_{0}^{1}\int_{0}^{1} x\exp(x-y+2z)\mathrm{d}x\mathrm{d}y\mathrm{d}z \\
    &= \frac{1}{M} \int_{0}^{1}\int_{0}^{1}-\exp(-y+2z)\mathrm{d}y\mathrm{d}z \\
    &= \frac{1}{M}\int_{0}^{1}\exp(2z-1)-\exp(2z) \mathrm{d}z \\
    &= \frac{1}{2M}(\mathrm{e} - \mathrm{e}^{-1} - \mathrm{e}^2 + 1) \\
    \hat{y} &= \frac{1}{M} \int_{0}^{1}\int_{0}^{1}\int_{0}^{1} y\exp(x-y+2z)\mathrm{d}x\mathrm{d}y\mathrm{d}z \\
    &= \frac{1}{M} \int_{0}^{1}\int_{0}^{1}y\exp(1-y+2z)-\exp(-y+2z)\mathrm{d}y\mathrm{d}z \\
    &= \frac{1}{M} \int_{0}^{1}(3\exp(2z)-2\exp(2z-1)-\exp(2z+1))\mathrm{d}z \\
    &= \frac{1}{2M}(3\mathrm{e}^2-\mathrm{e}^3+2\mathrm{e}^{-1}-3) \\
    \hat{z} &= \frac{1}{M} \int_{0}^{1}\int_{0}^{1}\int_{0}^{1} z\exp(x-y+2z)\mathrm{d}x\mathrm{d}y\mathrm{d}z \\
    &= \frac{1}{M} \int_{0}^{1} z(\exp(1+2z)+\exp(2z-1)-2\exp(2z))\mathrm{d}z \\
    &= \frac{1}{2M}(\mathrm{e}^3 + 2\mathrm{e} - 2\mathrm{e}^2+\mathrm{e}^{-1}-2) \\
\end{aligned}$$
Therefore, the center of mass $\vec{x} = $
$$\begin{bmatrix}
    \frac{\mathrm{e} - \mathrm{e}^{-1} - \mathrm{e}^2 + 1}{\mathrm{e}^3 - 2\mathrm{e}^2 - \mathrm{e}^{-1} + 2} \\
    \frac{3\mathrm{e}^2-\mathrm{e}^3+2\mathrm{e}^{-1}-3}{\mathrm{e}^3 - 2\mathrm{e}^2 - \mathrm{e}^{-1} + 2} \\
    \frac{\mathrm{e}^3 + 2\mathrm{e} - 2\mathrm{e}^2+\mathrm{e}^{-1}-2}{\mathrm{e}^3 - 2\mathrm{e}^2 - \mathrm{e}^{-1} + 2}
\end{bmatrix}$$
The above are not correct, but I am too tired to reconsider the new question...


\end{document}